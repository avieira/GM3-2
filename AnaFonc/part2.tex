\part{Espaces de Banach}
Soit E un espace vectoriel normé.
\Def{}{Un espace E normé est complet si chaque suite de Cauchy $\{u_n\}_{n\in\mathbb{N}}\subset E$ :
\[\forall \varepsilon >0, \exists N\in\mathbb{N}; \forall m>N, \forall k>N, ||u_m-u_k||<\varepsilon\]
converge dans E.}

\Def{}{Un espace vectoriel normé E est dit de Banach s'il est complet.}

\Exemp{}{Tous les espaces vectoriel de dimension finie sont de Banach (demonstration basé sur l'équivalence des normes)}

\Def{}{Soit $\{u_n\}_n\subset E$, avec E espace de Banach.\\
La série $\sum_n u_n$ converge normalement si et seulement si $\sum_n ||u_n|| <\infty$}

\Theo{}{Si $\sum_n u_n$ converge normalement alors $\sum_n u_n$ converge dans E, espace de Banach.}

\begin{dem}
	Il suffit de montrer que $\{\sum_{k=0}^n u_k\}_n$ est une suite de Cauchy.
	\begin{eqnarray*}
		\left\| \sum_{k=0}^{N_1} u_k - \sum_{k=0}^{N_2} u_k \right\| &=& \left\| \sum_{k=N_1+1}^{N_2} u_k \right\| \\
									  &\leq& \sum_{k=N_1+1}^{N_2} \| u_k \| \text{ convergente par hypothèse}
	\end{eqnarray*}
\end{dem}

\Theo{}{Soit X un expace métrique, E un espace de Banach. On définit : 
\[\mathcal{C}_{\infty}^b(X,E)=\{f:X\to E; f\in\mathcal{C}(X),\ f \text{ bornée}\}\]
Alors $\mathcal{C}_{\infty}^b(X,E)$ est un espace de Banach muni de la norme : 
\[||f||_{\infty}=\sup_{x\in X} ||f(x)||_E\]}

\begin{dem}
Démonstration à reprendre.
\end{dem}

\Theo{}{$L^p(X,\mathcal{B},\mu)$ où $(X,\mathcal{B},\mu)$ est un espace mesuré, $1\leq p < +\infty$ est un espace de Banach.}

\begin{dem}
Soit $\{f_n\}$ une suite de Cauchy dans $L^p$.
\begin{enumerate}
\item On va extraire une sous-suite de $\{f_n\}$ - notée $\{{f_n}_k\}$ - qui converge $\mu$pp.\\
On prend la sous-suite $\{{f_n}_k\}$ tel que :
\[\forall k, \|{f_n}_{k+1}-{f_n}_k\| \leq \frac{1}{2^{k+1}}\]
Pour construire $\{{f_n}_k\}$ : \\
$\{f_n\}$ est de Cauchy :
\[\forall \varepsilon>0, \exists N; \forall m,n\geq N, \|f_n-f_m\|<\varepsilon\]

Prenons $\varepsilon=\frac{1}{2}$, alors :
\[\exists N_{\frac{1}{2}}; \forall m,n\geq N, \|f_n-f_m\|_p<\varepsilon\]

\bigskip
Prenons $n_0=N_{\frac{1}{2}}$ ($\forall m\geq n_0, \|f_m-f_{n_0}\|$).\\
Ensuite, soit $\varepsilon=\frac{1}{4}$, alors 
\[\exists N_{\frac{1}{4}}; \forall m,n\geq N_{\frac{1}{4}}; \|f_m-f_n\|_p < \frac{1}{4}\]
Prenons $n_1=N_{\frac{1}{4}}$, et ainsi de suite. A la fin, on a $\{{f_n}_k\}$.

\bigskip
\item Démontrons que $\{{f_n}_k\}$ converge $\mu$pp.\\
Notons :
\[f_N(x)=f_{n_0}+\sum_{k=0}^{N-1} (f_{n_{k+1}}(x)-f_{n_k}(x))\]
\[h_N(x)=|f_{n_0}|+\sum_{k=0}^{N-1} |f_{n_{k+1}}(x)-f_{n_k}(x)|\]

On note $f(x)$ et $h(x)$ la limite de ces deux suites de fonction.

\bigskip
Pour $h_N(x)$, par l'inégalité triangulaire, on a :
\begin{eqnarray*}
	\|h_N\|_p&\leq& \|f_{n_0}\|_p + \sum_{k=0}^{N-1} \|f_{n_{k+1}}-f_{n_k}\|_p\\
		&\leq& \|f_{n_0}\|_p + \sum_{k=0}^{N-1} \frac{1}{2^{k+1}} \text{ par définition de } \{f_{n_k}\} \\
	 	&\leq& \|f_{n_0}\| + 2
\end{eqnarray*}
En utilisant le théorème de Beppo-Levi, on obtient que $h_N(x)\xrightarrow{\mu pp}h(x)\ \Rightarrow\ h(x)\in L^p(X,\mathcal{B},\mu)$

\bigskip
Comme $h_N$ est positive et croissante, $h_N$ converge normalement.\\
Donc $f_N(x)\xrightarrow{\mu pp} f(x)$ et $f(x)\in L^p(X,\mathcal{B},\mu)$ (car $\|f\|_p\leq \|h\|_p$).

\bigskip
\item $f_N(x)=f_{n_N}(x)$ (par définition de $f_N(x)$), donc la sous-suite $f_{n_k}$ converge $\mu$pp.

\bigskip
Considérons :
\begin{eqnarray*}
	|f-f_N|^p&\leq& 2^p \max \{|f_N(x)|^p,\ |f(x)|^p\} (|f-f_N|\leq 2\max\{|f_N|,|f|\})\\
			       &\leq& 2^p (|f_N(x)|^p + |f(x)|^p)\\
				&\leq& 2^{p+1} |h(x)|^p\in L^p
\end{eqnarray*}
car $|f_N(x)|\leq h(x)$ et $|f(x)|\leq h(x)$

alors par le théorème de la convergence dominée de Lebesgue :
\[f_N\to f \text{ dans } L^p\]

\bigskip
\item Démontrons que si $\{{f_n}_k\}$ converge dans $L^p(X,\mathcal{B},\mu)$ alors $\{f_n\}$ converge dans $L^p(X,\mathcal{B},\mu)$.
\[\|f-f_n\|_p\leq \|f-f_{n_k}\|_p + \|f_{n_k}-f_n\|_p\]
Comme $\{f_n\}$ est de Cauchy :
\[\forall \varepsilon>0; \exists N_{\varepsilon}; \forall n,m \geq N_{\varepsilon}, \|f_m-f_n\|<\varepsilon\]
Comme $f_n\to f$ dans $L^p$, alors $\forall \varepsilon>0, \exists M_{\varepsilon}; \forall n_k>M_{\varepsilon}$ :
\[\|f_{n_k}-f\|_p<\varepsilon\]
Si maintenant, $n_k>\max\{N_{\varepsilon}, M_{\varepsilon}\},\ n>N_{\varepsilon}$, on a :
\[\|f-f_n\|_p \leq 2\varepsilon\]
\end{enumerate}
\end{dem}

\Rem{}{$L^{\infty}(X,\mathcal{B},\mu)$ est aussi un espace de Banach.}

\Theo{de Banach-Steinhaus}{Soient L et E deux espaces de Banach et $\{T_n\}_{i\in I}$ une suite d'applications linéaires continues, $T_n:L\to E$. On suppose que $\forall x\in L, \sup_{i\in I} |T_i(x)|_E <+\infty$. Alors :
\[\sup_{i\in I} \|T_i\|_{L,E}<+\infty\]
(ie $\exists C>0; \forall x\in L, \forall i\in I, \|T_i(x)\|_E\leq C\|x\|_L$)}

\Rem{}{Ce théorème nous permet d'obtenir des estimations uniformes à partir d'estimations ponctuelles.}

\Lem{de Baire}{Soit X un espace métrique complet. Soit $\{X_n\}_{n\in\mathbb{N}}$ une suite de fermés dans X, et $\mathring{X_n}$.
\[\Rightarrow \overset{\circ}{\widehat{\bigcup_{n\in\mathbb{N}} X_n}}=\varnothing\]}

\begin{dem}
On définit $X_n=\{x\in L; \forall i\in I, \|T_(x)\|_E\leq n\}$
\begin{itemize}
	\item Il est clair que $X_n$ est un fermé.
	\item $\bigcup_{n=1}^{\infty} X_n=L$ car, $\forall x\in L,\ \sup_i \|T_i(x)\|< \infty$
\end{itemize}
Par le lemme de Baire, comme $\overset{\circ}{\widehat{\bigcup_{n\in\mathbb{N}} X_n}}=\mathring{L}\neq \varnothing$, alors $\exists n_0; \mathring{X_{n_0}}\neq \varnothing$
$\Rightarrow \exists x_0\in L; \exists r>0; B(x_0,r)\subset X_{n_0}$

Par définition de $X_{n_0}$, on a :
\[\|T_i(x_0+rz)\|\leq n_0,\ \forall i\in I,\ \forall z\in B(0,1)\]

Comme $T_i$ linéaire et par l'inégalité triangulaire :
\[r\|T_i(z)\|-\|T_i(x_0)\|\leq \|T_i(x_0+rz)\|\leq n_0\]

D'où :
\[\|T_i(z)\|\leq \frac{n_0+\|T_i(x_0)\|}{r}\]
\[\|T_i(z)\|\leq C \text{ pour } z\in B(0,1)\]
\end{dem}

\Coro{}{Soient L et E deux espaces de Banach. Soit $\{T_i\}$ une suite d'opérations linéaires continues, tel que \[T_i(x)\xrightarrow{i\to+\infty}T(x),\ \forall x\in L\]
Alors :
\begin{itemize}
	\item $\sup \|T_i\| <+\infty$
	\item $T\in\mathcal{L}_c(L,E)$
	\item $\|T\|\leq\lim \inf \|T_i\|$
\end{itemize}}

\begin{dem}
\begin{enumerate}
\item Vient directement du théorème
\item $\exists C>0; \|T_i(x)\|_E\leq X\|x\|_L; \forall i, \forall x\in L$
On passe à la limite :
\[\|T(x)\|_E\leq X\|x\|_L, \forall x\in L\]

Il est évident que T est linéaire $\Rightarrow\ T\in \mathcal{L}_C(L,E)$ et $\|T\|_{\mathcal{L}_C(L,E)}\leq \lim \inf \|T_i\|$
\end{enumerate}
\end{dem}

\Coro{}{Soient L et E des espaces de Banach, alors $L_c(L,E)$ est de Banach (muni d'une norme $\|\bullet\|_{L_C(L,E)}$)}

\begin{dem}
Soit $\{T_i\}$ une suite de Cauchy
\[\forall \varepsilon>0, \exists N_{\varepsilon}\in\mathbb{N}; \forall n,m\geq N_{\varepsilon}, \|T_n-T_m\|\leq \varepsilon\]

Alors la suite $\{T_n(x)\}$ est aussi de Cauchy $\forall x\in L$, car :
\[\|T_n(x)-T_m(x)\|_E\leq \|T_n-T_m\|\times \|x\|_L\leq \varepsilon \|x\|_L\]

Donc $\{T_n(x)\}$ converge ponctuellement vers une limite qu'on note $T(x)$. Donc, d'après le corollaire précédent, $T\in \mathcal{L}_C(L,E)$, ce qu'il fallait démontrer.
\end{dem}

\Coro{sur l'interpolation}{Soit $\{T^{(n)}\}$ une suite de subdivisions. $\exists f\in \mathcal{C}[0,1]$ telle que $\{L_n(f)\}_n$ (construit sur $\{T^{(n)}\}$) ne converge pas uniformément vers f.}

\begin{dem}
\[L_n(f,x)=\sum_{i=0}^n f(x_i)l_i(x)\]
avec \[l_i(x)=\prod_{j=0,j\neq i}^n \frac{x-x_j}{x_i-x_j}\]

Notons $\lambda_n(x)=\sum_{i=0}^n |l_i(x)|$ et $\lambda_n=\max_{x\in[0,1]} \lambda_n(x)$, appelée la constante de Lebesgue.\\
On a $L_n\in\mathcal{L}_C(\mathcal{C}[0,1],\mathcal{C}[0,1])$.
\[|L_n(f,x)\leq \|f\|_{\infty} \lambda_n\Rightarrow \|L_n\|\leq \lambda_n\]
On peut prouver que $\|L_n\|=\lambda_n$.

\bigskip
Pour la subdivision de Tchebychev :
\[\lambda_n > \frac{\ln(n)}{8\sqrt{\pi}}\]
Comme $\|L_n\|=\lambda_n>\frac{\ln(n)}{8\sqrt{\pi}}\xrightarrow[n\to+\infty]{}+\infty$, $\exists f\in\mathcal{C}[0,1]$ telle que le polynôme ne converge pas uniformément vers f.\\
(car si cela était le cas, on aurait une contradiction avec le premier corollaire).

\bigskip
Mais comme cette subdivision est optimale au sens de la norme uniforme, ceci est vrai pour toutes les subdivisions.
\end{dem}

\newpage
