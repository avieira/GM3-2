\documentclass{article}
\input{../preambule}
%\usepackage[8pt]{extsizes}

\hypersetup{colorlinks=true, urlcolor=bleu, linkcolor=red}

%Def = Definition
%Theo = Théorème
%Prop = Propriété
%Coro = Corollaire
%Lem = Lemme

\makeatletter
\@addtoreset{section}{part}
\makeatother

\begin{document}

\setcounter{tocdepth}{4}
\tableofcontents
\newpage

\part{Analyse Numerique}
\section{Interpolation polynomiale}
\begin{enumerate}
\item \textbf{Existence et unicité du polynôme d'interpolation}
\item \textbf{Base de Lagrange}
\item Formule du polynôme dans la base de Lagrange
\item \textbf{Base de Newton}
\item Formule du polynôme dans la base de Newton
\item Différence divisée
\item \textbf{Redéfnition du polynôme dans la base de Newton avec les différences divisées}
\item \textbf{Expression de l'erreur entre f et le polynôme d'interpolation}
\item Définition convergence méthode d'interpolation en un point
\item Lien entre subdivision et convergence
\item Définition d'un spline

\section{Intégration numérique}
\item Formule de Newton-Cotes
\item Définition degré d'exactitude
\item \textbf{Degré d'exactitude d'une formule de N-C}
\item Approximation avec méthode du point milieu
\item Théorème de la moyenne
\item \textbf{Calcul de l'erreur d'interpolation de la méthode du point milieu}
\item Formule de quadrature de la méthode du trapèze
\item \textbf{Erreur de la méthode}
\item Méthode de quadrature et erreur méthode de Simpson
\item Théorème si ordre n pair ou impair
\item Définition de convergence et de stabilité
\item \textbf{Équivalence à stabilité}
\item Reprise des méthodes précédentes en composite
\item Méthode de Runge
\item \textbf{Degré maximal de degré d'exactitude}
\item Définition polynômes orthogonaux, polynôme unitaire
\item Nombre de racines d'un polynôme orthogonal de degré n
\item Méthode de quadrature de Gauss

\section{Equations différentielles}
\item Définition erreur de quadrature
\item Définition consistance
\item Équivalence à consistance
\item Définition de stabilité
\item Condition nécessaire pour que la méthode soit stable
\item Définition de convergence
\item Condition nécessaire pour la convergence
\item Définition de l'ordre d'une méthode
\item Lien consistance et ordre
\item Méthode de Runge-Kutta : définition de la méthode, présentation sous forme d'un tableau
\item Condition pour consistance de Runge-Kutta
\item Théorème sur calcul des $y_k$ de Runge-Kutta et sur sa stabilité
\item Condition sur la méthode d'ordre 2
\item Méthodes à pas liés : expression des méthodes, condition sur $\alpha_0$
\item Méthodes d'Adams : ordre de Bashforth ? Moulton ?
\item Principe de Prédiction-Correction ?
\item Définition de consistance. Equivalence à consistance.
\item Définition de stabilité. Equivalence à stabilité.
\item Définition de convergence. Equivalence à convergence. 
\end{enumerate}

\newpage
\part{Analyse Fonctionnelle}
\section{Espaces normés}
\begin{enumerate}
	\item Définition norme, espace normé
	\item Normes en dimension finie
	\item Définition de continue
	\item Définition de limite dans un espace normé
	\item Définition d'application borné
	\item \textbf{Théorème équivalence pour application linéaire}
	\item Norme pour ensemble des applications linéaires
	\item \textbf{3 propriétés de cette norme}
	\item \textbf{Si A linéaire, alors $A^{-1}$ ?}
	\item Condition pour $A^{-1}$ continue
\section{Espaces compacts}
	\item Définition espace compact
	\item Théorème de Bolzano-Weierstrass
	\item \textbf{Citer un compact de $\mathbb{R}^n$ ?}
	\item \textbf{Si X est compact ?}
	\item \textbf{Si X compact, $f: X\to \mathbb{R}$, alors ?} (x2)
\section{Espaces de Banach}
	\item Définition espace complet
	\item Définition espace de Banach
	\item Espaces vectoriels de dimension finie ?
	\item Définition de convergence normale d'une série
	\item \textbf{Lien entre convergence normale et convergence dans un espace de Banach}
	\item $\mathcal{C}_{\infty}^b(X,E)$, $L^p(X,\mathcal{B},\mu),\ 1\leq p\leq\infty$
	\item \textbf{Suite d'applications linéaires continues entre deux espaces de Banach, que dire sur la norme des $T_n$ ?}
	\item \textbf{Corollaire avec convergence des $T_i$}
	\item \textbf{Corollaire : $L_C(L,E)$ ?}
	\item \textbf{Corollaire sur l'interpolation}
\section{Espaces de Hilbert}
	\item Définition du produit scalaire et espace hermitien/pré-hilbertien
	\item \textbf{Inégalité de Cauchy-Schwartz}
	\item \textbf{Inégalité de Minkowsky}
	\item Définition d'espace de Hilbert
	\item Produit scalaire sur $L_2(X,\mathcal{B},\mu)$
	\item Définition d'espace convexe
	\item \textbf{Théorème de projection}
	\item \textbf{Corollaire sur norme des projections}
	\item \textbf{Projection sur un sous-espace fermé}
	\item Corollaire sev supplémentaires
	\item Définition dense (avec définition adhérence)
	\item Définition famille totale
	\item Équivalence totale
	\item Définition d'un dual
	\item \textbf{Théorème de Riesz-Fréchet}
	\item Définition de base hilbertienne
	\item Théorème sur décomposition dans un espace de Hilbert
	\item Définition de l'adjoint
	\item \textbf{Unicité}
	\item Définition de auto-adjoint
	\item Définition de vecteur propre, ensemble résolvant, spectre
	\item Valeurs propres dans les cas finis ou infinis
	\item Définition de relativement compact, opérateur relativement compact
	\item Théorème sur base hibertienne et vecteurs propres
\section{TD}
	\item Equivalence espace préhilbertien
\end{enumerate}

\newpage
\part{Distributions}
\section{Théorie générale}
\begin{enumerate}
\item Ensemble et convergence sur $\mathcal{D}$
\item Ensemble et convergence sur $\mathcal{D}'$
\item Ensemble des fonctions localement intégrables
\item Égalité des distributions
\item Dérivée d'une distribution
\item Dérivée d'une suite de distribution et convergence
\item Application aux séries
\item Dérivées généralisées
\item Définition d'absolument convergent
\item Conséquence sur la dérivée
\item CNS sur la convergence d'une suite/série de distributions
\end{enumerate}

\section{Équations différentielles et intégrales - Produit de convolution - Calcul symbolique}
\begin{enumerate}
\item Dérivée nulle
\item Défintion de f et g convolables. CS d'existence et $\in L^1$
\item Produit tensoriel de 2 distributions
\item Support d'une fonction / Support d'une ditribution
\item Théorème de prolongement
\item Élément neutre de la convolution. Lien avec la dérivation.
\item Définition de $\mathcal{D}'_g$. Lien avec la convolution.
\item Formulaire pour le calcul symbolique
\end{enumerate}

\section{Transformation de Fourier}
\begin{enumerate}
\item Définition Transformation de Fourier
\item 4 propriétés de la transformée de Fourier
\item Transformée de la fonction dérivée
\item Ensemble de espace topologique de $\mathcal{S}$
\item Stabilité par la transformation de Fourier
\item Espace des distributions tempérées
\item TF des distributions tempérées
\item Continuité de la TF
\item Définition ditribution produit
\item Définition : Translation d'une distribution
\item Transformée d'une distribution translatée
\item Définition $f_{\alpha}$ et transformée. Cas particulier.
\item Transformée de la Dirac et de 1
\item Formule de réciprocité
\item Définition $T_{\lambda x}$
\item Théorème si S distribution à support compact
\item Théorème sur la transformée d'une convoluée de distributions
\end{enumerate}

\section{Distributions périodiques - Série de Fourier}
\begin{enumerate}
\item Définition distribution périodique
\item Définition peigne de Dirac
\item Théorème sur distributions périodiques
\end{enumerate}

\newpage
\part{Gleyse}
\begin{enumerate}
	\item Problème bien posé, stabilité
	\item Deix définition d'approché, entier maximal, sur combien de bits sont codés les entiers ? 
	\item Passer de binaire à hexa et à entier, et inversement. Passer d'un réel à son codage en binaire.
	\item Passer d'un entier positif à négatif et inversement, opération sur les entiers.
	\item Nombre de bits mantisse et exposant SP et DP, limite pour M et E, calcul de l'exposant. \\
		Nombres flottants caractéristiques, ordre de grandeur. \\
		Erreur relative d'affectation, Hypothèse de Vignes et de Hamming, \\
		Erreur sur opérations flottantes.
	\item Comment mettre en place CESTAC ? Utilité ? Avec arrondi ou troncature ? Sur quoi on agit ? Gaussien ?
	\item Définition norme vectorielle induite, conditionnement, démonstration sur variations sur premier ou second membre.
	\item Définition de l'ordre de convergence, rapport avec les dérivées. Si r=1, vitesses linéaires et logarithmiques.
	\item 
\end{enumerate}

\newpage
\part{Probabilités}
\section{Convergences}
\begin{enumerate}
\item Convergence en probabilité et presque sûr
\item Lien série convergence presque sûr
\item Définition covariance
\item Inégalité de Cauchy-Schwarz - équivalence sur égalité ce $cov^2$
\item Loi faible et forte des grands nombres
\item Convergence en loi
\item Lien entre différentes convergences
\item Théorème central limit
\end{enumerate}
\section{Vecteurs aléatoires}
\begin{enumerate}
\item Définition espérance vecteur aléatoire
\item Matrice de variance - covariance
\item Définition des vecteurs aléatoires gaussiens
\item Loi de $Z=X+Y$, $X$ indépendant de $Y$
\end{enumerate}
\section{Fonctions caractéristiques}
\begin{enumerate}
\item Définition de la fonction caractéristique
\item Si $X\hookrightarrow \mathcal{N}(0,1)$
\item 5 propriétés (linéarité, de même avec des vecteurs, rapport avec complémentaire, si X var intégrable, et X et Y indépendantes)
\item Transformée de Laplace
\end{enumerate}
\section{Conditionnement - Espérance conditionnelle}
\begin{enumerate}
\item Théorème de Doob - Loi conditionnelle
\item 3 cas où q identifiable : si $\mu_X$ discrète, si (X,Y) admet une densité, si X et Y sont indépendantes.
\item Espérance conditionnelle si admet une loi conditionnelle
\end{enumerate}

\part{Statistiques}
\begin{enumerate}
\section{Introduction}
	\item Définition : Espace des observations, échantillon d'une loi P
	\item Définition densité jointe
	\item Définition modèle statistique
	\item Définition statistique
	\item Moyenne et variance emprique. \textbf{Espérance et variance de la première, esperance de la deuxième}.
	\item Définition d'une fonction génératrice des moments
	\item Définition famille exponentielle
	\item Théorème sur somme des composantes de échantillon de famille exponentielle
	\item Théorème de Slutsky
\section{Exhaustivité}
	\item 
\end{enumerate}


\newpage
\part{Equations Différentielles}
\section{Existence, unicité, régularité}
\begin{enumerate}
\item Problème de Cauchy
\item \textbf{Réécriture du problème de Cauchy}
\item Lipschitzienne
\item \textbf{Lemme : CS de lpschitzienne}
\item \textbf{Théorème de Cauchy-Lipschitz}
\item Solution maximale, globale
\item Définition de $z(t,x_0)$
\item \textbf{Théorème sur équation satisfaite par z}
\section{Transformations}
\item Définition de $\gamma_t(x_0)$
\section{Équations linéaires}
\item \textbf{Solution de $\dot{x}=Ax$}
\item \textbf{Si $\tilde{A}=TAT^{-1}$, alors $e^{\tilde{A}t}=?$}
\section{Resolution des systèmes}
\item \textbf{Si $y\in\mathbb{C}^n$ solution d'un système, que dire des parties réelles et imaginaires ?}
\item Forme des solutions ?
\section{Stabilité}
\item Equivalence à la stabilité asymptotique
\item Définition d'un point déquilibre
\item Solution passant par un point d'équilibre
\item Définition de stabilité
\item Définition d'asymptotiquement stable
\item Théorème de Liapounov
\item De même pour la non stabilité
\item Définition de (semi-)positive/négative
\item Définition de dérivée orbitale
\item Théorème sur fonction de Liapounov et stabilité puis localement asymptotiquement stable
\section{Systèmes hamiltoniens}
\item Définition d'un système hamiltonien et de l'hamiltonien
\item Définition d'intégrale première
\item Equivalence intégrale première
\item Rapport avec l'hamiltonien ?
\item Stabilité et hamiltonien positif/négatif
\item $H=\frac{1}{2}p^2+\phi(q)$ : Equivalence à point d'équilibre? Stabilité ?
\item Deux derniers résultats
\section{Complément Master 2}
\item Théorème de Carathéodory
\item Définition Variété, thm de Whitney
\item Def Groupe de Lie
\item Flow Box Theorem
\item Définition solution périodique, orbite
\item Proposition pour $x_0\in\Gamma$ et solution périodique
\item Thm de Bendixon
\item Thm de Poincaré-Bendixon
\end{enumerate}

\newpage
\part{Programmation objet}
\begin{enumerate}
\item Définition de classe, attributs, opérations, message, objet, réïfication.
\item Représentation UML de classes, public, protected, private.
\item Relation de composition. Représentation UML.
\item Relation de visibilité. Représentation UML.
\item Relation d'héritage. Représentation UML. Héritage simple et multiple.
\item Diagramme de séquence
\item Contexte
\item Cycle de vie en cascade, en V, en spirale, itératif
\item 3 principes du génie logiciel, lien avec la programmation par objets
\item Sous-systèmes, BPM, user case, diagramme d'activité, paquetage
\end{enumerate}

\newpage
\part{Java}
\section{Notations}
\begin{enumerate}
\item Créer un objet avec sa classe
\item Accéder à une variable d'instance d'un objet
\item Envoi d'un message à <o>
\end{enumerate}
\section{Java}
\begin{enumerate}
\item Définir une classe
\item Création d'un objet
\item Compilation
\end{enumerate}
\end{document}
