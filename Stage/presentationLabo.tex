\section{Présentation du laboratoire}
Ce stage a été effectué au LITIS, le Laboratoire d'Informatique, du Traitement de l'Information et des Systèmes, qui est impliqué dans trois principaux établissement d'enseignement supérieur :
\begin{itemize}
	\item l'Université de Rouen
	\item l'Université du Havre
	\item l'INSA de Rouen
\end{itemize}

\bigskip
D'après son \href{http://www.litislab.eu/}{site internet}, la recherche dans ce laboratoire s'organise en trois axes :
\begin{itemize}
	\item l'axe Combinatoire et algorithmes
	\item l'axe Interaction des systèmes complexes
	\item l'axe Traitement des masses de données
\end{itemize}

\bigskip
C'est dans ce dernier que s'est déroulé mon stage, plus précisément dans l'équipe "Document et Apprentissage". Cette équipe compte 16 membres permanents, dont mon maître de stage Stéphane \bsc{Canu}. \\
Cette équipe a pour \enquote{objectif [...] l'étude des techniques de modélisation et d’apprentissage statistiques permettant d’appréhender la diversité des données [...] et la nature des solutions attendues}. Elle a de nombreux partenaires, aussi bien universitaires (Université du Kent, UK - Université Cornell, USA) qu'industriels (Bertin Techonolgies, Thalès, OrangeLabs). 

\newpage
