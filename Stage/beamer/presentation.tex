\documentclass{beamer}

\usepackage[frenchb]{babel}
\usepackage[T1]{fontenc}
\usepackage[utf8x]{inputenc}
\usepackage{minted}
 
\usetheme{Berkeley}
\usecolortheme{seahorse}
\useinnertheme{rounded}

\title[Soutenance de stage]{Soutenance de stage : \\Travail sur les algorithmes de recommandation permettant de convertir les données disponibles en un conseil personnalisé utile}
\author{Alexandre \bsc{Vieira}}
\institute{INSA de Rouen}
\date{15 octobre 2013}


\AtBeginSection[]
{
	\begin{frame}
		\frametitle{Sommaire}
		\tableofcontents[currentsection, hideothersubsections]
	\end{frame}
}

\begin{document}

\begin{frame}
\titlepage
\end{frame}

\begin{frame}
	\frametitle{Sommaire}
	\tableofcontents
\end{frame}

\section[Laboratoire]{Présentation du laboratoire}
 
\begin{frame}
	\frametitle{LITIS}
	\begin{itemize}
		\item Présentation générale du laboratoire
		\item Axes de recherche
		\item Équipe "Document et Apprentissage"
		\item Partenaires nombreux et variés
	\end{itemize}
\end{frame}

\section[Théorique]{Explications théoriques}
\subsection{SVM linéaire}
\begin{frame}
	\frametitle{Les SVM linéaires : le cas séparable}
	\includegraphics[scale=0.5]{../graph/graphSep.png}
\end{frame}

\subsection[Noyaux]{Introduction des noyaux}
\begin{frame}
	\frametitle{Les noyaux : transformation de l'espace}
	\begin{center}\includegraphics[scale=0.65]{../graph/graphNoyau.png}\end{center}
\end{frame}

\subsection[CV]{Validation croisée}
\begin{frame}
	\frametitle{Recherche des hyperparamètres}
	\begin{center}\includegraphics[scale=0.65]{../graph/regionParam.png}\end{center}
\end{frame}

\section[Travail effectué]{Travail effectué : Exploitation de données}
\subsection[Cadre]{Explication du cadre de recherche}
\begin{frame}
	\frametitle{Explication du cadre de recherche}
	\begin{itemize}
		\item En relation avec un doctorant de l'ENSICAEN
		\item Type de données
		\item But de cette recherche
	\end{itemize}
\end{frame}

\subsection[Dvlppmt]{Développement des solutions mises en place}
\begin{frame}
	\frametitle{Séparation directe}
Reconnaissance du sexe :
	\begin{tabular}{|*{6}{c|}}
	\hline
	 & $P_1$ & $P_2$ & $P_3$ & $P_4$ & $P_5$ \\
	\hline
	Mêmes personnes & 69\% & 75\% & 69\% & 69\% & 73\% \\
	\hline
	Séparation personnes & 46\% & 52\% & 40\% & 55\% & 53\% \\
	\hline
	\end{tabular}
\end{frame}

\begin{frame}
	\frametitle{Séparation des cas 1 ou 2 mains}
Reconnaissance du sexe pour une main :
	\begin{tabular}{|*{6}{c|}}
	\hline
	 & $P_1$ & $P_2$ & $P_3$ & $P_4$ & $P_5$ \\
	\hline
	Mêmes personnes & 65\% & 75\% & 68\% & 74\% & 75\% \\
	\hline
	Séparation personnes & 53\% & 60\% & 44\% & 55\% & 56\% \\
	\hline
	\end{tabular}

\bigskip
Reconnaissance du sexe pour deux mains :
	\begin{tabular}{|*{6}{c|}}
	\hline
	 & $P_1$ & $P_2$ & $P_3$ & $P_4$ & $P_5$ \\
	\hline
	Mêmes personnes & 70\% & 75\% & 75\% & 81\% & 79\% \\
	\hline
	Séparation personnes & 41\% & 51\% & 49\% & 53\% & 51\% \\
	\hline
	\end{tabular}
\end{frame}

\begin{frame}
	\frametitle{Élimination des variables}
Reconnaissance du sexe pour une main :
	\begin{tabular}{|*{6}{c|}}
	\hline
	 & $P_1$ & $P_2$ & $P_3$ & $P_4$ & $P_5$ \\
	\hline
	Mêmes personnes & 71\% & 74\% & 67\% & 68\% & 75\% \\
	\hline
	Séparation personnes & 55\% & 57\% & 39\% & 60\% & 58\% \\
	\hline
	\end{tabular}

\bigskip
Reconnaissance du sexe pour deux mains :
	\begin{tabular}{|*{6}{c|}}
	\hline
	 & $P_1$ & $P_2$ & $P_3$ & $P_4$ & $P_5$ \\
	\hline
	Mêmes personnes & 76\% & 77\% & 72\% & 78\% & 72\% \\
	\hline
	Séparation personnes & 39\% & 50\% & 47\% & 57\% & 53\% \\
	\hline
	\end{tabular}
\end{frame}

\begin{frame}
	\frametitle{Majority Vote}
	\begin{itemize}
		\item Méthode de boosting
		\item Résultats prometteurs
	\end{itemize}
	\begin{center}\begin{tabular}{|c|c|}
	\hline
	Nombre de mains & 99\% \\
	\hline
	Genre & 48\% \\
	\hline
	\end{tabular}\end{center}
\end{frame}

\section[Conclusion]{Perspectives et conclusion}
\begin{frame}
	\frametitle{Perspectives}
	\begin{itemize}
		\item Vérifier le boosting avec le genre
		\item Choix différent des classifieurs dans le boosting %Voir par rapport aux 10 vecteurs pour une personne pour faire vote majoritaire
		\item Utilisation d'autres méthodes de boosting
	\end{itemize}
\end{frame}

\begin{frame}
	\frametitle{Conclusion}
	\begin{itemize}
		\item Stage m'ayant appris énormément
		\item Approche du monde de la recherche
		\item Utile pour mon projet professionnel
	\end{itemize}
\end{frame}

\appendix
\section{Formulation du problème primal et dual}
\begin{frame}
	\frametitle{Cas linéaire}
	Primal SVM :
	\[\left\{\begin{array}{c c c}
	\min_{\omega, b, \xi} & \frac{1}{2} \|\omega\|^2 +\frac{C}{d} \sum_{i=1}^n \xi_i^d &\\
	\text{avec } & y_i(\omega^Tx_i+b)\geq 1-\xi_i & i=1,n\\
		& \xi_i\geq 0 & i=1,n
	\end{array}\right.\]
	Dual SVM :
	\[\left\{\begin{array}{l l}
	\min_\alpha & \frac{1}{2} \alpha^TG\alpha -e^T\alpha\\
	\text{avec } & y^T\alpha=0\\
		& \alpha_i\geq 0, \hspace{0.5cm} i=1,n
	\end{array}\right.\]

\end{frame}
\end{document}
