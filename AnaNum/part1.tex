\part{Interpolation polynomiale}
\section*{Introduction}

A partir de $\{x_j,y_j\}_{j=0}^n$, on veut trouver un polynôme $L_n(x)$ tel que $L_n(x_j)=y_j$, pour j allant de 0 à n.

\bigskip
\begin{itemize}
\item Le polynôme est-il bien posé ? (existence, unicité)
\item Construction du polynôme
\item Calculer l'erreur d'interpolation
\item La suite $\{L_n(f,x)\}_n$ convergera-t-elle vers f ?
\item Quelle sera la meilleure subdivision ?
\end{itemize}

\section{Existence et unicité}
\Theo{Existence et unicité du polynôme d'interpolation}{Il existe un unique polynôme d'interpolation $L_n(x)$ tel que $L_n(x_j)=y_j$, j=0..n, si et seulement si $\{x_j\}_{j=0}^n$ sont tous distincts.}

\begin{dem}
\[L_n(x)=\sum_{j=0}^n a_j x^j\]

En effet, $\text{deg}(L_n(x))\leq n$.\\
On a de plus :
\[L_n(x_k)=y_k, k=0..n\]
A partir de cela, on peut construire un système de n+1 équations d'inconnues $a_0,a_1,...,a_n$ ainsi :
\[\left\{
\begin{array}{c c c}
a_0 + a_1 x_0 + ... + a_n x_0^n &=& y_0 \\
a_0 + a_1 x_1 + ... + a_n x_1^n &=& y_1 \\
\vdots &\vdots& \vdots \\
a_0 + a_1 x_n + ... + a_n x_n^n &=& y_n
\end{array}
\right.
 \Leftrightarrow
\begin{pmatrix}
  1    &   x_0   &   \cdots   &   x_0^n  \\
  1    &   x_1   &   \cdots   &   x_1^n  \\
\vdots &  \vdots &   \ddots   &   \vdots \\
  1    &   x_n   &   \cdots   &   x_n^n 
\end{pmatrix}
\begin{pmatrix}
a_0 \\
a_1 \\
\vdots \\
a_n
\end{pmatrix}
=
\begin{pmatrix}
y_0 \\
y_1 \\
\vdots \\
y_n
\end{pmatrix}
\]
D'où : 
\[
\text{Le choix de } (a_0,...,a_n) \text{ est unique} \Leftrightarrow \begin{tabular}{|c c c c c|}
1 & $x_0$ & $x_0^2$ & $\cdots$ & $x_0^n$\\
\vdots & \vdots & \vdots & \vdots & \vdots\\
\vdots & \vdots & \vdots & \vdots & \vdots\\
1 & $x_n$ & $x_n^2$ & $\cdots$ & $x_n^n$ \\
\end{tabular} \neq 0 \]

On doit démontrer pour cela : \[\begin{vmatrix}
1 & x_0 & \cdots & x_0^n \\
1 & x_1 & \cdots & x_1^n \\
\vdots & \vdots & \ddots & \vdots \\
1 & x_n & \cdots & x_n^n 
\end{vmatrix}
 = \prod_{0\leq j<i \leq n} (x_i - x_j)\]

On le démontre par récurrence. Au rang n=1 : \[\begin{vmatrix} 1 & x_0 \\ 1 & x_1 \end{vmatrix} = x_1 - x_0\]
On suppose l'égalité vraie au rang n.

\begin{eqnarray*}
\begin{vmatrix}
1 & x_0 & \cdots & x_0^{n+1} \\
1 & x_1 & \cdots & x_1^{n+1} \\
\vdots & \vdots & \ddots & \vdots \\
1 & x_{n+1} & \cdots & x_{n+1}^{n+1} 
\end{vmatrix}
&=& 
\begin{vmatrix}
1 & x_0-x_{n+1} & \cdots & x_0^{n+1}-x_{n+1}x_0^n \\
1 & x_1-x_{n+1} & \cdots & x_1^{n+1}-x_{n+1}x_1^n \\
\vdots & \vdots & \ddots & \vdots \\
1 & 0 & \cdots & 0 
\end{vmatrix} \\
&=&(-1)^{n+1} \begin{vmatrix}
x_0-x_{n+1} & \cdots & x_0^{n+1}-x_{n+1}x_0^n \\
x_1-x_{n+1} & \cdots & x_1^{n+1}-x_{n+1}x_1^n \\
\vdots      & \ddots &         \vdots         \\
x_n-x_{n+1} & \cdots & x_n^{n+1}-x_{n+1}x_n^n \\
\end{vmatrix} \\
&=&(-1)^{n+1} \begin{vmatrix}
x_0-x_{n+1} & \cdots & (x_0-x_{n+1})x_0^n \\
x_1-x_{n+1} & \cdots & (x_1-x_{n+1})x_1^n \\
\vdots      & \ddots &         \vdots         \\
x_n-x_{n+1} & \cdots & (x_n-x_{n+1})x_n^n \\
\end{vmatrix}  \\ 
&=& (-1)^{n+1} \prod_{j=0}^n (x_j - x_{n+1})  \prod_{0\leq j<i \leq n} (x_i - x_j) \\
&=& ((-1)^{n+1})^2 \prod_{j=0}^n (x_{n+1} - x_j)  \prod_{0\leq j<i \leq n} (x_i - x_j) \\
&=& \prod_{0\leq j<i \leq n+1} (x_i - x_j)
\end{eqnarray*}
D'où :
\begin{eqnarray*}
\text{Le choix de } (a_0,...,a_n) \text{ est unique} &\Leftrightarrow&  \prod_{0\leq j<i \leq n} (x_i - x_j) \neq 0\\
&\Leftrightarrow& \forall i\neq j, x_i - x_j \neq 0 \\
&\Leftrightarrow& (x_0,...,x_n) \text{ tous distincts}
\end{eqnarray*}
\end{dem}

\section{Construction du polynôme d'interpolation}
\subsection{Construction à partir de la base de Lagrange}
\Propo{Base de Lagrange}{La famille de polynôme $\{l_i\}_{i=0}^n$ définie par :
\[\forall j\neq i, l_i(x_i)=1,\ l_i(x_j)=0\]forment une base de $P_n$, espace vctoriel des polynômes de degré au plus n
}

\begin{dem}
Prouvons tout d'abord que le système est libre. Soit $(\lambda_0,...,\lambda_n)\in\mathbb{R}^n$. On cherche :
\[\forall x\in \mathbb{R}, \lambda_0l_0(x) + ... + \lambda_n l_n(x) = 0\]
Or, pour $x=x_i$, on trouve directement $\lambda_i=0,\ \forall 1\leq i\leq n$.

De plus :  \begin{eqnarray*}
\text{card}(\{l_i\}_{i=0}^n) &=& n+1 \\
\dim (P_n) &=& n+1
\end{eqnarray*}
La famille est donc une base de $P_n$
\end{dem}

\paragraph{Recherche du polynôme : \\}
$l_i(x)$ admet n racines. D'où : \[\forall x\in \mathbb{R},\  l_i(x)=\alpha_i \prod_{\underset{j\neq i}{j=0}}^n (x-x_j)\]

En posant $x=x_i$ : \[\alpha_i \prod_{\underset{j\neq i}{j=0}}^n (x_i-x_j)=1 \Leftrightarrow \alpha_i = \frac{1}{\prod_{\underset{j\neq i}{j=0}}^n (x-x_j)}\]
D'où :
\[l_i(x)=\prod_{\underset{j\neq i}{j=0}}^n \frac{x-x_j}{x_i-x_j}\]

On cherche à présent $L_n(f,x)$. D'après ce qu'on vient de démontrer, on peut avoir : 
\[\forall x\in \mathbb{R},\ L_n(x)=\sum_{i=0}^n a_i l_i(x)\]

$x=x_j \Rightarrow f(x_j)=a_jl_j(x_j)=a_j$ d'où :

\bigskip
\Formu{Polynôme d'interpolation dans la base de Lagrange}{
\[L_n(f,x) = \sum_{i=0}^n f(x_i) \prod_{\underset{j\neq i}{j=0}}^n \frac{x-x_j}{x_i-x_j}\]
}

\begin{rmq}
Si on ajoute un point il devient difficile de l'intégrer au calcul
\end{rmq}

\subsection{Construction à partir de la base de Newton}
\Propo{Base de Newton}{La famille $\{N_i\}_{i=0}^n$ de polynôme telle que : \[N_0(x)=1,\ \forall 1\leq i \leq n-1,\ N_{i+1}(x)= \prod_{j=0}^i (x-x_j)\]
est une base de $P_n$}

\begin{dem}
De même que pour la base de Lagrange, il suffit de montrer que la famille est libre, que cardinal et dimension sont encore une fois égaux (et finis).
Soit $(\lambda_0,...,\lambda_n)\in\mathbb{R}^n$. On cherche :
\[\forall x\in \mathbb{R}, \lambda_0 +\lambda_1(x-x_0) +... + \lambda_n \prod_{j=0}^{n-1} (x-x_j) = 0\]
\begin{eqnarray*}
x=x_0 &\Rightarrow& \lambda_0 = 0 \\
x=x_1 &\Rightarrow& \underbrace{\lambda_0}_{=0} + \lambda_1(x_1-x_0) + \lambda_1(x_1-x_0)\underbrace{(x_1-x_1)}_{=0}+... = \lambda_1\underbrace{(x_1-x_0)}_{\neq 0} = 0 \\
      &\Rightarrow& \lambda_1 = 0 \\
      &\vdots& \\
x=x_n &\Rightarrow& \lambda_n = 0
\end{eqnarray*}

Donc la famille est bien une base de $P_n$
\end{dem}

\Formu{Polynôme d'interpolation dans la base de Newton}{Le polynôme d'interpolation dans la base de Newton est définie récursivement par la formule : 
\[L_n(x)=L_{n-1}(x)+\left(f(x_n)-L_{n-1}(x_n)\right)\frac{\omega_{n-1}(x)}{\omega_{n-1}(x_n)}\]
Avec $\omega_0(x)=x-x_0$ et $\omega_{n-1}(x)=\prod_{j=0}^{n-1} (x-x_j)$}

\begin{dem}
On va chercher les racines du polynôme $L_n(x)-L_{n-1}(x)$.
\[L_n(x_n)-L_{n-1}(x_n) \neq 0\]
\[\forall 0\leq i \leq n-1,\ L_n(x_i) - L_{n-1}(x_i) = f(x_i)-f(x_i) = 0\]
D'où $x_0,...,x_{n-1}$ racines du polynôme. On a donc : 
\[L_n(x) - L_{n-1}(x)=c_n \prod_{i=0}^{n-1} (x-x_i) = c_n \omega_{n-1}(x)\]
Déterminons à présent $c_n$ à partir du point $x_n$ : 
\begin{eqnarray*}
L_n(x_n)-L_{n-1}(x_n) &=& f(x_n) - L_{n-1}(x_n) \\
		      &=& c_n \underbrace{\omega_{n-1}(x_n)}_{\neq 0} \\
\\
\Rightarrow c_n = \frac{f(x_n)-L_{n-1}(x_n)}{\omega_{n-1}(x_n)}
\end{eqnarray*}
\end{dem}

Intéressons-nous de plus près au calcul de $c_n$.
\begin{eqnarray*}
c_n &=& \frac{f(x_n)}{\omega_{n-1}(x_n)} - \sum_{i=0}^{n-1} f(x_i) \prod_{\underset{j\neq i}{j=0}}^{n-1} \frac{x_n-x_j}{(x_i-x_j)(x_n-x_j)(x_n-x_i)} \\
    &=& \frac{f(x_n)}{\omega_{n-1}(x_n)} - \sum_{i=0}^{n-1} f(x_i) \prod_{\underset{j\neq i}{j=0}}^{n-1} \frac{1}{(x_i-x_j)(x_n-x_i)} \\
    &=& \frac{f(x_n)}{\omega_{n-1}(x_n)} + \sum_{i=0}^{n-1} f(x_i) \prod_{\underset{j\neq i}{j=0}}^{n-1} \frac{1}{(x_i-x_j)(x_i-x_n)} \\
    &=& \frac{f(x_n)}{\omega_{n-1}(x_n)} + \sum_{i=0}^{n-1} f(x_i) \prod_{\underset{j\neq i}{j=0}}^n \frac{1}{x_i-x_j} \\
    &=& \sum_{i=0}^n f(x_i) \prod_{\underset{j\neq i}{j=0}}^n \frac{1}{x_i-x_j}
\end{eqnarray*}

On a donc \[L_n(x) - L_{n-1}(x) = \omega_{n-1}(x) \sum_{i=0}^n f(x_i) \prod_{\underset{j\neq i}{j=0}}^n \frac{1}{x_i-x_j}\]
Pour faciliter le calcul de $\sum_{i=0}^n f(x_i) \prod_{\underset{j\neq i}{j=0}}^n \frac{1}{x_i-x_j}$, on introduit les différences divisées.

\Def{Différence divisée de f}{\begin{itemize}
\item Ordre 0 : $f[x_0]=f(x_0)$
\item Ordre 1 : $\underset{x_i \neq x_j}{f[x_i,x_j]} = \frac{f(x_j)-f(x_i)}{x_j-x_i}$
\item $\vdots$
\item Ordre k : $f[x_i,...,x_{i+k}]  \frac{f[x_{i+k},...,x_{i+1}]-f[x_i,...,x_{i+k-1}}{x_{i+k}-x_i}$
\end{itemize}}

\Lem{Différence divisée et polynôme de Newton}{\[\sum_{i=0}^n \frac{f(x_i)}{\prod_{\underset{j\neq i}{j=0}}^n (x_i-x_j)} = f[x_1,...,x_n]\]}
\begin{dem}
On démontre ce lemme par récurrence : \\
$n=0 \Rightarrow f[x_0]=f(x_0)$\\
Supposons à présent que \[f[x_0,...,x_n]=\sum_{i=0}^n \frac{f(x_i)}{\prod_{\underset{j\neq i}{j=0}}^n (x_i-x_j)},\text{ pour } (x_0,...,x_n) \text{ tous distincts.}\]
Prouvons le à présent au rang n+1 en utilisant la définition de la différence divisée :
\[f[x_0,...,x_{n+1}]=\frac{f[x_1,...,x_{n+1}]-f[x_0,...,x_n]}{x_{n+1}-x_0}\]
avec $x_{n+1}$ distinct des autres points. \\
Par hypothèse de récurrence, on a :
\[f[x_1,...,x_{n+1}]=\sum_{i=1}^{n+1} \frac{f(x_i)}{\prod_{\underset{j\neq i}{j=1}}^{n+1} (x_i-x_j)}\]
\begin{eqnarray*}
f[x_0,...,x_{n+1}]&=&\frac{1}{x_{n+1}-x_0} \left[\sum_{i=1}^{n+1} \frac{f(x_i)}{\prod_{\underset{j\neq i}{j=1}}^{n+1} (x_i-x_j)}-\sum_{i=0}^n \frac{f(x_i)}{\prod_{\underset{j\neq i}{j=0}}^n (x_i-x_j)} \right]\\
		&=& \frac{1}{x_{n+1}-x_0}\left[\frac{f(x_{n+1})}{\prod_{j=1}^{n}(x_{n+1}-x_j)} - \frac{f(x_{0})}{\prod_{j=1}^{n}(x_0-x_j)}+\sum_{i=1}^n\left(\frac{f(x_i)}{\prod_{\underset{j\neq i}{j=1}}^{n+1} (x_i-x_j)}-\frac{f(x_i)}{\prod_{\underset{j\neq i}{j=0}}^n (x_i-x_j)} \right) \right] \\
		&=& \frac{f(x_{n+1})}{\prod_{j=0}^{n}(x_{n+1}-x_j)} + \frac{f(x_{0})}{\prod_{j=1}^{n+1}(x_0-x_j)} + \frac{1}{x_{n+1}-x_0} \sum_{i=1}^n\left(\frac{f(x_i)}{\prod_{\underset{j\neq i}{j=0}}^{n+1} (x_i-x_j)}(x_i-x_0-x_i+x_{n+1}) \right) \\
		&=& \sum_{i=0}^{n+1} \frac{f(x_i)}{\prod_{\underset{j\neq i}{j=0}}^{n+1} (x_i-x_j)}
\end{eqnarray*}
\end{dem}

\Formu{Interpolation dans la base de Newton}{\[L_n(x)=\sum_{j=0}^n f[x_0,...,x_j] \prod_{i=0}^{j-1}(x-x_i)\]}

\section{Convergence, étude d'erreur}
\Theo{}{Soient $x_0,...,x_n$ distincts, $f\in\mathcal{C}^{n+1}$ sur $[\min\{x_0,...,x_n\},\max\{x_0,...,x_n\}]$ et $L_n(f,x)$ son polynôme d'interpolation.
\[\forall x\in\left[\min\{x_0,...,x_n\},\max\{x_0,...,x_n\}\right],\exists \zeta \in \left]\min\{x_0,...,x_n\},\max\{x_0,...,x_n\}\right[ \]
\[f(x)-L_n(f,x)=\frac{f^{(n+1)}(\zeta)}{(n+1)!}\omega_n(x)\]}

\begin{dem}
	On définit une fonction \[g(s)=f(s)-L_n(f,s)-k\omega_n(s)\]
	La constante k est choisie est telle que $g(x)=0$, c'est-à-dire :
	\[k=\frac{f(x)-L_n(f,x)}{\omega_n(x)},\ x\neq x_i, i=0..n\]
	
	Donc g(s) s'annule aux points $x_1$,...,$x_n$,$x$ (car $\forall i$, $g(x_i)=\underbrace{f(x_i)-L_n(f,x_i)}_{=0}-K\underbrace{\omega_n(x_i)}_{=0}$)\\

	Donc g(s) admet n+2 zéros, donc, par le théorème de Rolle, g' admet n+1 zéros.\\
	On continue, et on obtient que $g^{(n+1)}$ a un zéro. 

	\[\exists \zeta_x ; g^{(n+1)}(\zeta_x)=0\]
	\[g^{(n+1)}(s)=f^{(n+1)}(s)-\underbrace{L_n^{(n+1)}(f,s)}_{=0 (deg n)}-k\omega_n^{(n+1)}(s)\]
	Or $(\omega_n(s))^{(n+1)}=(X^{n+1}+...)^{(n+1)}=(n+1)!$. D'où : 
	\[g^{(n+1)}(s)=f^{(n+1)}(s)-\frac{f(x)-L_n(f,x)}{\omega_n(x)}\omega_n(s)\]
	\[\Rightarrow f^{(n+1)}(\zeta_x)-\frac{f(x)-L_n(f,x)}{\omega_n(x)}\omega_n(\zeta_x)=0\]
	D'où le résultat.
	
\end{dem}

\Coro{}{L'erreur dépend de la subdivision $\{x_i\}_{i=0}^n$ choisie.}

\begin{rmq}
\begin{enumerate}
\item Divergence du polynôme d'interpolation (?)
\item Une meilleure approche peut être obtenue par un changement de subdivision
\item Vu qu'il n'existe pas de subdivision pour laquelle le polynôme d'interpolation converge pour toute les fonctions, on peut diviser l'intervalle en petits sous-intervalles et interpoler avec des polynômes de degré plus petit.
\item On peut imposer les conditions de raccord : condition de régularité aux points communs de polynômes différents. Cela débouche sur la notion de splines : à partir de $\{x_i\}_{i=0}^n$ on construit une fonction cubique par morceaux qui interpole la fonction initiale. 
\end{enumerate}
\end{rmq}

\Def{}{La méthode d'interpolation converge vers $f$ au point $x^*\in[a,b]$ si : \[\lim_{n\to+\infty} L_n(f,x^*)=f(x^*)\]}

\Theo{}{Quelque soit la subdivision d'intervalle [a,b], il existe une fonction $f\in\mathcal{C}[a,b]$ tel que $L_n(f,x)\not\to f(x)$}
\Theo{}{$\forall f\in\mathcal{C}[a,b]$, il existe une subdivision de [a,b] tel que la suite correspondante du polynôme d'interpolation converge vers f.}

\section{Splines}
\subsection{Généralités}
$a=x_0<...<x_n=b$
\Def{}{La fonction $S_m(x)$ est dite spline de degré m si :
	\begin{enumerate}
		\item Sur chaque $[x_i,x_{i+1}]$, i=0 à n+1, $S_m(x)$ est un polynôme de degré au plus n
		\item $S_m(f,x)\in\mathcal{C}^{m-1}[a,b]$ (régularité)
	\end{enumerate}}

Le spline est un spline d'interpolation si $\forall i$ de 0 à n, $f(x_i)=S_m(f,x_i)$.\\
Sinon, on parle de spline d'approximation.

\subsection{Splines cubiques}
Sur chaqu $[x_i,x_{i+1}]$, on interpole f par une fonction cubique $S_i(x)$. Avec les conditions de régularité : 
\begin{enumerate}
	\item $S_i(x_i)=S_{i-1}(x_i)$ (i=1 à n-1)
	\item $S_i(x_i)=f(x_i)$ (i=0 à n)
	\item $S_i'(x_i)=S_{i-1}'(x_i)$ (i=1 à n-1)
	\item $S_i''(x_i)=S_{i-1}''(x_i)$ (i=1 à n-1)
\end{enumerate}

Cela nous fait $4n-2$ équations sur les $s_i(x)$, i=0 à n-1.

Or, s$_i(x)=a_ix^3+b_ix^2+c_ix+d_i$, donc pour définir tous les $\{s_i\}_{i=0}^{n-1}$, il nous faut 4n conditions. Il nous manque 2 conditions manquantes.

\bigskip
\textbf{Possibilités}
\begin{enumerate}
	\item $S_0''(x_0)=0$ et $S_{n-1}''(x_n)=0$. (Spline naturel)
	\item Si on connaît $f'(x_0)$ et $f'(x_n)$ alors $S_0'(x_0)=f'(x_0)$ et $S_{n-1}'(x_n)=f(x_n)$
	\item \begin{itemize}
			\item Interpoler f aux points $x_0,x_1,x_2$ par $L_2(f,x)$ puis $S_0''(x_0)=L_2(f,x_0)''$
			\item Interpoler f aux points $x_{n-2},x_{n-1},x_n$ par $\bar{L}_2(f,x)$ puis imposer $S_{n-1}''(x_n)=L_2(f,x_n)''$
		\end{itemize}		
\end{enumerate}
