\documentclass{article}
\input{../../preambule}

\hypersetup{colorlinks=true, urlcolor=bleu, linkcolor=red}

%Def = Definition
%Theo = Théorème
%Prop = Propriété
%Coro = Corollaire
%Lem = Lemme

\begin{document}

\begin{tabular}{c c c c c}
	$f(x_{k-1})$ &            & & & \\
	             & $\searrow$ & & & \\
	             &            & $\frac{f(x_k)-f(x_{k-1})}{h}$ & & \\
	             & $\nearrow$ &                               & $\searrow$ & \\
	$f(x_k)$     &            &                               &            & $\frac{f(x_{k+1})-2f(x_k)+f(x_{k-1})}{2h^2}$\\
	             & $\searrow$ &                               & $\nearrow$ &  \\
	             &            & $\frac{f(x_{k+1})-f(x_k)}{h}$ &            &  \\

	             & $\nearrow$ &                               & & \\
	$f(x_{k+1})$& & & & 
\end{tabular}\\

D'où \:
\[L_2(f)(x)=f(x_{k-1})+\frac{f(x_k)-f(x_{k-1})}{h}(x-x_{k-1})+\frac{f(x_{k+1})-2f(x_k)+f(x_{k-1})}{2h^2}(x-x_{k-1})(x-x_k)\]

\[\int_{x_k}^{x_{k+1}} L_2(f)(t)dt = f(x_{k-1})h + \frac{f(x_{k-1})}{h}\left( (\underbrace{x_{k+1}-x_{k-1}}_{=2h})^2-(\underbrace{x_k-x_{k-1}}_{=h})^2\right)+\frac{f(x_{k+1})-2f(x_k)+f(x_{k-1})}{2h^2}\times (c)\]

\begin{eqnarray*}
	(c)=\int_{x_k}^{x_{k+1}}(x-x_{k-1})(x-x_k)dx &=& \int_0^h u(u+h) du \text{ en posant } u=x-x_k \\
						    &=&\frac{1}{3}h^3 +\frac{1}{2}h^3 \\
						    &=&\frac{5}{6} h^3
\end{eqnarray*}

\begin{eqnarray*}
\Rightarrow \int_{x_k}^{x_{k+1}} L_2(f)(t)dt &=& f(x_{k-1})h + f(x_{k-1})\frac{3}{2}h+\frac{5}{12}h(f(x_{k+1})-2f(x_k)+f(x_{k-1})) \\
					 &=& \frac{h}{12}(-f_{k-1}+8f_k +5f_{k+1})
\end{eqnarray*}
\end{document} 
