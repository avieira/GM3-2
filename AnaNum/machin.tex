\documentclass{article}
\input{../../preambule}

\hypersetup{colorlinks=true, urlcolor=bleu, linkcolor=red}

%Def = Definition
%Theo = Théorème
%Prop = Propriété
%Coro = Corollaire
%Lem = Lemme

\begin{document}

A est bornée $\Rightarrow$ A est continue

A est bornée, donc : \[\exists C; \forall x_1, x_2 \in E, \|A(x_2-x_1)\| = \|Ax_2 -Ax_1\| \leq C\|x_2 - x_1\|\]

On cherche à montrer que :
\[\forall x_1\in E,\ \forall \varepsilon >0,\ \exists \delta>0;\ \forall x_2\in E;\ \|x_2-x_1\|\leq \delta \Rightarrow \|Ax_2 - Ax_1\| \leq \varepsilon \]

Prenons $\delta=\frac{\varepsilon}{C}$. Ainsi :
\[\|x_2-x_1\|\leq \frac{\varepsilon}{C} \Leftrightarrow C\|x_2-x_1\|\leq \varepsilon\]

Or :
\[\forall x_1, x_2 \in E, \|A(x_2-x_1)\| = \|Ax_2 -Ax_1\| \leq C\|x_2 - x_1\|\]

D'où :
\[\|x_2-x_1\|\leq \delta \Rightarrow \|Ax_2 - Ax_1\| \leq \varepsilon\]

\end{document}
