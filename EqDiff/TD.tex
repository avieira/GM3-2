\documentclass{article}
\input{../../preambule}

\hypersetup{colorlinks=true, urlcolor=bleu, linkcolor=red}

%Def = Definition
%Theo = Théorème
%Prop = Propriété
%Coro = Corollaire
%Lem = Lemme

\makeatletter
\@addtoreset{section}{part}
\makeatother

\begin{document}

\setcounter{tocdepth}{4}
\tableofcontents
\newpage

\paragraph{Exercice 1 : \\}
\begin{enumerate}
\item Résoudre $(4x+xy^2)dx+(y+x^2y)dy=0$. Trouvez $y(x)$ telle que $y(1)=2$
\item Résoudre $\frac{dy}{dx}=\frac{\tan x}{\cos y}$
\item Résoudre $x(y^2-1)dx+y(x^2-1)dy=0$
\end{enumerate}

\paragraph{Exercice 2 : \\}
\begin{enumerate}
\item Résoudre $(3x^2+y\cos x)dx + (\sin x-4y^3)dy=0$
\item Résoudre $\frac{xy+1}{y}dx+\frac{2y-x}{y^2}dy=0$
\end{enumerate}

\paragraph{Exercice 3 : \\}
\begin{enumerate}
\item Prouvez que si \[\frac{\frac{\partial P}{\partial y}-\frac{\partial Q}{\partial x}}{Q}=F(x)\]
alors $\mu(x)=\exp\left(\int F(x)dx\right)$ est un facteur intégrant.
\item Résoudre $(2xy+y^4)dx-(2x^2-xy^3)dy = 0$ 
\end{enumerate}

\paragraph{Exercice 4 : \\}
\begin{enumerate}
\item Résoudre $\frac{dy}{dx} + 2xy = xe^{-x^2}$
\item Résoudre $\frac{dy}{dx} + M(x)y = N(x)$
	\begin{itemize}
	\item Avec la même méthode que précédemment
	\item En utilisant un facteur intégrant
	\end{itemize}
\end{enumerate}

\paragraph{Exercice 5 : Equation de Bernoulli \\}
\begin{enumerate}
	\item \[\frac{dy}{dx} + P(x)y = Q(x)y^s,\ s\neq\{0,1\}\]
		On pose $z(x)=(y(x))^{1-s}$.\\
		Transformez l'équation de Bernoulli par $z(x)$
	\item Résoudre \[3xy' - y = 3x\ln|x| y^4\]
	\item Résoudre \[y' + \frac{2y}{x} = \frac{2\sqrt{y}}{\cos^2 x}\] 
\end{enumerate}

\paragraph{Exercice 6 : Equation de Riccati \\}
\begin{enumerate}
\item \[\frac{dy}{dx}=M(x)y^2 + N(x)y + P(x)\]
	Cette équation n'a pas forcément de solutions, et on ne connaît aucun moyen de résoudre cette équation.\\
	Cependant, si on a $y_p$ une solution particulière de l'équation, on peut la résoudre.\\
	On pose \[y-y_p = \frac{1}{z}\]
	Exprimez l'équation de Riccati en utilisant $z$.
\item Résoudre \[(1-x^3)y' + x^2y + y^2 -2x=0\]
\item Résoudre le problème de Cauchy \[\frac{dx}{dt} = x^2,\ \in\mathbb{R}\] avec $x(0)=x_0$
\item Résoure \[\frac{dx}{dt}=x^{\frac{1}{3}},\ x\in\mathbb{R}\] avec $x(0)=0$
\end{enumerate}

\end{document}
