\documentclass{article}

\usepackage[utf8x]{inputenc}
\usepackage[T1]{fontenc}
\usepackage[francais]{babel}
\usepackage{lmodern}
\usepackage{amsthm}
\usepackage{amsmath}
\usepackage{amssymb}
\usepackage{mathrsfs}
\usepackage{verbatim}
\usepackage{moreverb}
\usepackage{listings}
\usepackage{graphicx}
\usepackage{hyperref}
\usepackage{minted}

\hypersetup{colorlinks=true, linkcolor=red}

\author{Conrad \bsc{Hillairet} et Alexandre \bsc{Vieira}}
\title{Projet P3 : Analyse Numérique \\ \Large{Méthode d'Adams-Bashforth-2}}
\date{\today}

\begin{document}
\maketitle

\setcounter{tocdepth}{4}
\tableofcontents
\newpage

\section*{Introduction}
Les équations différentielles sont utiles dans de nombreux domaines scientifiques. On les utilise bien souvent en physique, en chimie ou même dans la finance. \\
Bien qu'on sache comment résoudre des équations différentielles, toutes ne sont pas forcément résolvable avec les connaissances actuelles. Pour cela, on cherche à avoir une solution approchée de ces équations via des méthodes itératives. Il en existe plusieurs sortes :\\
\begin{itemize}
	\item Les méthodes à un pas, où chaque itération ne dépend que de la précédente. On peut citer comme exemple la méthode d'Euler
	\item Les méthodes à pas liés, dont les suites récurrentes sont au moins d'ordre deux.
\end{itemize}

\bigskip
La méthode d'Adams-Bashforth-2, développée dans ce rapport, est une méthode à pas liés. Elle est définie ainsi :
\[
	\left\{
		\begin{array}{c c c}
			y_{t+1}&=&y_t + \frac{h}{2}(3f_t-f_{t-1})\\
			y(x_0)&=&y_0
		\end{array}
	\right.
\]

où :
\[f_t=f(x_t,y_t)\]
\[y_t=y(x_t)\]

Ce rapport se partagera ainsi :
\begin{itemize}
	\item Nous commencerons par présenter la méthode d'Euler et la méthode d'Adams-Bashforth-2, ainsi que différents théorèmes
	\item Nous présenterons ensuite différentes implémentations en C
	\item Nous testerons enfin nos codes sur des exemples que nous commenterons.
\end{itemize}

\section{Présentation des méthodes d'Euler et d'Adams-Bashforth-2}
\subsection{Méthode d'Euler}
Cette méthode est utilisée pour initialiser la méthode d'Adams. Elle a en effet besoin de deux points à chaque itération, or, le problème de Cauchy n'en donne qu'un seul.
\subsubsection{Définition de la méthode}
Pour définir la méthode d'Euler, repartons du système différentiel, qu'on définit sur $[x_t,x_{t+1}]$, $t\in\mathbb{N}$, intervalle de longueur h :
\[\left\{ \begin{array}{c c c}
	dy = f(x,y) dx \\
	y(x_t)=y_t
\end{array}\right.\]
Ce système équivaut à :
\[\int_{y_t}^{y_{t+1}} dy = \int_{x_t}^{x_{t+1}} f(x,y) dx\]
On approche la deuxième intégrale par la méthode des rectangles à gauche :
\[\int_{x_t}^{x_{t+1}} f(x,y) dx \approx \underbrace{(x_{t+1}-x_t)}_{=h}f(x_t,y_t)\]

Ainsi :
\[y_{t+1} \approx y_t + hf(x_t,y_t)\]

\subsubsection{Stabilité et convergence}
\begin{itemize}
	\item Si $f$ est lipschitzienne par rapport à y, la méthode est stable.
	\item La méthode est convergente.
\end{itemize}

\subsection{Méthode d'Adams-Bashforth-2}
\subsubsection{Définition de la méthode}

L'idée de cette méthode est simple : on interpole à l'ordre 1 la fonction $f(x,y)$ entre $x_{t-1}$ et $x_t$. Ainsi :
\begin{eqnarray*}
f(x,y) &\approx& f(x_t,y_t)+\frac{f(x_{t-1},y_{t-1})-f(x_t,y_t)}{x_{t-1}-x_t}(x-x_t)\\
       &\approx& f_t + \frac{f_{t-1}-f_t}{x_{t-1}-x_t}(x-x_t)
\end{eqnarray*}
On pose $L_2(f,x)=f_t + \frac{f_{t-1}-f_t}{x_{t-1}-x_t}(x-x_t)$.

\bigskip
On repart à présent du système différentiel de départ, qu'on définit entre $x_t$ et $x_{t+1}$. On considère qu'entre chaque $x_i$ et $x_{i+1}$, on a un pas constant $h$ :
\[ \left\{ \begin{array}{c c c}
	\frac{dy}{dx} &=& f(x,y) \\
		y(x_t)&=&y_t
\end{array}\right.\]

Ce système équivaut à : 
\[y_{t+1}=y_t + \int_{x_t}^{x_{t+1}} f(x,y) dx\]
On utilise ici l'approximation de $f$. On calcule pour cela $L_2(f,x_t)$ et $L_2(f,x_{t+1})$.
\begin{eqnarray*}
	L_2(f,x_t)&=&f_t\\
	L_2(f,x_{t+1})&=&f_t + \frac{f_{t-1}-f_t}{\underbrace{x_{t-1}-x_t}_{=-h}}\underbrace{(x_{t+1}-x_t)}_{=h}\\
			&=&2f_t - f_{t-1} 
\end{eqnarray*}

D'où :
\begin{eqnarray*}
	\int_{x_t}^{x_{t+1}} f(x,y)dx &\approx& \int_{x_t}^{x_{t+1}} L_2(f,x) dx \\
				      &\approx& \frac{h}{2}(L_2(f,x_{t+1}) + L_2(f,x_t))\\
			       	      &\approx& \frac{h}{2}(3f_t - f_{t-1})
\end{eqnarray*}

D'où la méthode d'Adams-Bashforth-2 :
\[y_{t+1}=y_t + \frac{h}{2}(3f_t - f_{t-1})\]

\subsubsection{Convergence de la méthode}
Démontrons qu'il existe $\eta$ tel que : 
\[\varepsilon(t,h)=\frac{5}{12} h^3 y^{(3)}(\eta)\]
Où $\varepsilon(t,h)$ est l'erreur sur la méthode.\\
Considérons d'abord l'approximation de la solution au point t :

\begin{eqnarray*}
	y(t) &\approx& y(t-h) + \frac{h}{2} [3f(t-h,y(t-h)) - f(t-2h,y(t-2h))] \\
	     &\approx& y(t-h) + \frac{h}{2} [3y'(t-h) - y'(t-2h)]
\end{eqnarray*}

On utilise à présent la formule de Taylor :
\begin{eqnarray*}
	y(t-h)&=&y(t)-hy'(t)+\frac{h^2}{2}y''(t)-\frac{h^3}{6}y^{(3)}(\eta)\\
       y'(t-h)&=&y'(t)-hy''(t)+\frac{h^2}{2}y^{(3)}(\eta)\\
      y'(t-2h)&=&y'(t)-2hy''(t)+2h^2y^{(3)}(\eta)
\end{eqnarray*}

On a donc :
\begin{eqnarray*}
	\varepsilon(t,h)&=&y(t)-y(t-h)-\frac{h}{2}[3y'(t-h)-y'(t-2h)]\\
			&=&hy'(t)-\frac{h^2}{2}y''(t)+\frac{h^3}{6}y^{(3)}(\eta)
	-\frac{3h}{2}y'(t) + \frac{3h^2}{2}y''(t) - \frac{3h^3}{4}y^{(3)}(t)
	+\frac{h}{2}y'(t)-h^2y''(t)+h^3y^{(3)}(\eta)\\
			&=&hy'(t)\left( 1-\frac{3}{2}+\frac{1}{2}\right) + h^2y''(t)\left( -\frac{1}{2} + \frac{3}{2}-1\right) 
	+ h^3y^{(3)}(\eta)\left(\frac{1}{6}-\frac{3}{4}+1\right) \\
			&=&\frac{5}{12}h^3y^{(3)}(\eta)
\end{eqnarray*}
Donc la méthode converge.

\subsubsection{Stabilité de la méthode}
Si f est lipschitzienne, alors la méthode converge.\\
Définissons pour cela deux méthodes :
\[\left\{ \begin{array}{c c c}
	y_{j+1} &=& y_j + \frac{h}{2}[3f(x_j,y_j)-f(x_{j-1},y_{j-1})]\\
	y(x_0) &=& y_0
\end{array}\right.\]
\[\left\{ \begin{array}{c c c}
	z_{j+1} &=& z_j + \frac{h}{2}[3f(x_j,z_j)-f(x_{j-1},z_{j-1})]+ \varepsilon_j\\
	z(x_0) &=& z_0
\end{array}\right.\]

\begin{eqnarray*}
	|y_{j+1}-z_{j+1}| &\leq& |y_j-z_j| + \frac{h}{2}[3|f(x_j,y_j)-f(x_j,z_j)| + |f(x_{j-1},y_{j-1})-f(x_{j-1},z_{j-1})|]+|\varepsilon_j| \\
	&\leq& |y_j-z_j| + \frac{b-a}{2}[3M|y_j-z_j| + M|y_{j-1}-z_{j-1}|]+|\varepsilon_j| \text{ car } f \text{ lipschitzienne} \\
	&\leq& C(|y_j-z_j|+ |y_{j-1}-z_{j-1}| + |\varepsilon_j|) \text{ avec } C = \frac{3}{2}(b-a)M+1 > 0 \\
	&\leq& C((1+C)|y_{j-1}-z_{j-1}| + |y_{j-2}-z_{j-2}| + |\varepsilon_j| + |\varepsilon_{j-1}|) \\
	&\leq& C'(|y_{j-1}-z_{j-1}| + |y_{j-2}-z_{j-2}| + |\varepsilon_j| + |\varepsilon_{j-1}|) \\
	&\leq& \vdots \\
	&\leq& C''\left(|y_1-z_1| + |y_0-z_0| + \sum_{i=1}^j |\varepsilon_i|\right) \\
\end{eqnarray*}

Or, on a $|y_1-z-1|\leq C(|y_0-z_0| + |\varepsilon_0|)$ car la méthode est stable. Ainsi :
\[|y_{j+1}-z_{j+1}|\leq C^{(3)}\left(|y_0-z_0| + \sum_{i=0}^j |\varepsilon_i|\right)\]

D'où :
\[\max_{0\leq j\leq n} |y_j - z_j| \leq C^{(3)}\left(|y_0-z_0| + \sum_{i=0}^{n-1} |\varepsilon_i|\right)\]

\bigskip
On peut également utiliser le théorème suivant : \\
Une méthode est stable si les racines du polynôme $\rho(\lambda)=0$ sont à l'intérieur du cercle unité ou elles sont simples si elles sont sur le cercle unité du plan complexe.\\
Le polynôme $\rho(\lambda)$ est défini par : \[\rho(\lambda)=\sum_{i=1}^k \alpha_i \lambda^i\]
où les $\alpha_i$ sont les coefficients devant les $y_{n-i}$

\bigskip
Le polynôme est dans notre cas : \[\lambda^2-\lambda=\lambda(\lambda-1)\]
Les racines sont donc 0 et 1, qui sont simples. La méthode est donc bien stable.

\subsubsection{Consistance de la méthode}
On utilise pour cela le théorème suivant :
Si on vérifie les conditions suivantes :
\[\sum_{i=1}^k \alpha_i = 0 \text{ et } \sum_{i=1}^k i\alpha_i + \beta_i =0\]
où $\alpha_i$ est définie comme précédemment et les $\beta_i$ sont les coefficients devant les $\beta_i$, alors la méthode est consistante. 

\bigskip
Dans notre exemple, on a la méthode : 
\[y_{t+1}=y_t + \frac{h}{2}(3f_t - f_{t-1})\]
D'où :
\[\alpha_1 = 1,\ \alpha_2=-1\]
et :
\[\beta_1 = \frac{3}{2},\ \beta_2=-\frac{1}{2}\]

On a donc : $\sum \alpha_i = 0$ et :
\[\sum i\alpha_i + \beta_i = 1-2+\frac{3}{2}-\frac{1}{2} = 0\]

La méthode est donc consistante. \\
Comme on l'a retrouvé précédemment, la méthode est bien convergente, car elle est consistante et stable.

\section{Présentation du code C}
Tout ce projet a été écrit en C. Nous présentons ici les différents codes écrits pour implémenter les différentes méthodes.\\
Nous avons également écrit un script bash ainsi qu'un script gnuplot pour automatiser le lancement des différents calculs. Ils seront présenter en Annexe.
\subsection{Code pour la méthode d'Euler}
\begin{minted}{c}
void euler(int n,double h,double *V0,double *V1,double *F)
{
	int i;

	for(i=0;i<n;i++)
		V1[i]=V0[i]+h*F[i];
}
\end{minted}

\subsection{Code pour la méthode d'Adams-Bashforth}
\begin{minted}{c}
void adamsBashforth(int n,double h,double *V1,double *V,double *F0,double *F1)
{
	int i;

	for(i=0;i<n;i++)
		V[i]=V1[i]+h/2.*(3*F1[i]-F0[i]);
}
\end{minted}

\subsection{main}
\begin{minted}{c}
void main()
{
	int n;
	double h;
	double *V0;
	double *V1;
	double *V;
	double *F0;
	double *F1;
	double xmax;
	int nmax;
	int i,j;

	//Recuperation des informations generales sur la methode
	scanf("%d %lf %lf",&n,&xmax,&h);

	//Creation des tableaux dynamiques
	V0=(double *) malloc(n*sizeof(double));
	V1=(double *) malloc(n*sizeof(double));
	V=(double *) malloc(n*sizeof(double));
	F0=(double *) malloc(n*sizeof(double));
	F1=(double *) malloc(n*sizeof(double));

	//Lecture et ecriture des conditions initiales
	for(i=0;i<n;i++)
	{
		scanf("%lf",&V0[i]);
		printf("%lf ",V0[i]);
	}
	printf("\n");

	//Calcul du nombre d'iterations
	nmax=trunc((xmax-V0[0])/h);

	//Initialisation avec la methode d'Euler explicite
	f(V0,F0);
	euler(n,h,V0,V1,F0);
	for(i=0;i<n;i++)
		printf("%lf ",V1[i]); 
	printf("\n");

	//Methode d'Adams-Bashforth
	f(V1,F1);
	for(i=2;i<(nmax+1);i++)
	{
		adamsBashforth(n,h,V1,V,F0,F1);
		for(j=0;j<n;j++)
		{
			printf("%lf ",V[j]);
			V0[j]=V1[j];
			V1[j]=V[j];
			f(V0,F0);
			f(V1,F1);
		}
		printf("\n");
	}
}
\end{minted}

\section{Présentation des différents exemples développés}
\subsection{Exemple introductif}
On commence par un exemple extrêmement simple : 
\[\dot{y}=y\]
\[y(0)=1\]
La solution est ici évidente : $y(x)=e^x$. \\
Cet exemple nous sert surtout à comparer la méthode d'Adams-Bashforth-2 avec celle d'Euler, et de voir la précision qu'on a pour un pas donné. Dans cet exemple, nous avons pris un pas de 0.01.

\begin{center} \includegraphics{../exempleIntro/image.eps} \end{center}

On voit très bien que pour un même pas, la méthode d'Adams-Bashforth-2 est déjà plus précise que celle d'Euler. La courbe obtenue par cette méthode s'approche énormément de la solution exacte, alors qu'on voit déjà que la méthode d'Euler s'en éloigne dans l'intervalle étudié.\\
Voyons maintenant dans d'autres exemples comment fonctionne cette méthode.

\subsection{Premier exemple}
Dans cet exemple, l'équation était la suivante :
\[y'=\sin(x) - \cos(y), y(0)=1\]



\begin{center}
	\includegraphics{../exemple1/image3.eps}
\end{center}

On remarque tout d'abord que :
\begin{itemize}
	\item On a un certain écart entre un pas de 0.1 et un pas de 0.01 pour la méthode d'Adams-Bashforth-2. Néanmoins, les solutions numériques obetnues semblent converger vers une même représentation graphique. 
	\item Cet exemple illustre aussi bien le fait que la méthode d'Adams-Bashforth soit d'ordre 2 et la méthode d'Euler soit d'ordre 1. En effet, on observe un écart non négligeable entre la méthode d'Euler et de la méthode d'Adams-Bashforth pour un pas de 0.1. Notons toutefois que cet écart diminue avec le pas.
\end{itemize}

\subsection{Deuxième exemple}
L'équation était :
\[xdx - ydy = yx^2dy - xy^2dx\]
\[y(1)=1\]

On transforme l'équation pour avoir :
\[\frac{dy}{dx}=\frac{x(1+y^2)}{y(1+x^2)}\]

Essayons de résoudre l'équation :
\begin{eqnarray*}
	\frac{y}{1+y^2}dy &=& \frac{x}{1+x^2} dx \\
	\frac{1}{2} \ln(1+y^2) &=& \frac{1}{2} \ln(1+x^2) + k \\
	y^2&=&C(1+x^2)-1
\end{eqnarray*}

Or, on veut $y(1)=1$, d'où $1=2C-1 \Rightarrow C=1$\\
On a donc $y^2=x^2$

\begin{center}
\includegraphics{../exemple2/image.eps}
\end{center}

\begin{itemize}
	\item La courbe ressemble beaucoup à ce qu'on doit obtenir (une droite affine). La convergence semble donc assurée.
	\item Avec des pas différents (0.1, 0.01 et 0.0001), la courbe était exactement la même.
\end{itemize}

\subsection{Troisième exemple}

Nous avions un système d'équation de la forme :
\[\left\{ \begin{array}{c c c}
y_1' &=& y_2\\
y_2' &=& -5\sin(y_1)\\
y_1(0) &=& \theta_0 \\
y_2(0) &=& 0
\end{array}\right.\]

On a pris pour cela deux valeurs de $\theta_0$ : 0.5 et 1.\\
Nous avons décidé sur cet exemple de comparer les résultats obtenus par la méthode d'Adams-Bashforth et par la méthode d'Euler. 

\subsubsection{Résultats avec $\theta_0=1$}
\bigskip
En commençant avec la méthode d'Adams-Bashforth-2, on voit que la métode semble converger, en réduisant le pas, vers une certaine solution. Avec un pas de 0.1, on semble quand même s'éloigner assez rapidement de la solution :
\begin{center}
\includegraphics{../exemple3/theta1/image.eps}\\
\includegraphics{../exemple3/theta1/imageAB.eps}
\end{center}

En comparaison, la méthode d'Euler semble bien moins puissante : elle ne donne pas une bonne solution approchée avec un pas de 0.1 :
\begin{center}
\includegraphics{../exemple3/theta1/imageEul.eps}
\end{center}

Si on compare à présent dirrectement la méthode d'Euler et la méthode d'Adams-Bashforth-2, la méthode d'Adams-Bashforth semble bien plus sûre que la méthode d'Euler, comme on peut le voir sur ces graphiques :

\begin{center}
\includegraphics{../exemple3/theta1/imageAB-Eul0-01.eps}\\
\includegraphics{../exemple3/theta1/imageAB-Eul0-001.eps}
\end{center}

\subsubsection{Résultats avec $\theta_0=0.5$}
Les résultats ne changent pas vraiment ici :
\begin{itemize}
	\item La convergence est toujours la même avec la méthode d'Adams-Bashforth-2 :
\begin{center}
\includegraphics{../exemple3/theta0/image.eps}\\
\includegraphics{../exemple3/theta0/imageAB.eps}
\end{center}

	\item La méthode d'Euler semble toujours moins puissante et n'est toujours pas digne de confiance avec un pas de 0.1 :
\begin{center}
\includegraphics{../exemple3/theta0/imageEul.eps}
\end{center}

	\item Si on compare les deux méthodes, la méthode d'Adams est toujours plus sûre que la méthode d'Euler :
\begin{center}
\includegraphics{../exemple3/theta0/imageAB-Eul0-01.eps}\\
\includegraphics{../exemple3/theta0/imageAB-Eul0-001.eps}
\end{center}
\end{itemize}

\subsection{Quatrième exemple}

Nous avions un système d'équation de la forme :
\[\left\{ \begin{array}{c c c}
y_1' &=& -2y_1 - 998 y_2\\
y_2' &=& -1000y_2\\
y_1(0) &=& 2\\
y_2(0) &=& 1
\end{array}\right.\]

On peut facilement calculer la solution de ce système.
Directement, on voit que :
\[y_2(t)=e^{-1000t}\]
Il nous reste donc à résoudre :
\[y_1'(t)=-2y_1(t)-998e^{-1000t}\]
Solution de l'équation homogène :
\[y_{1_H}(t)=Ce^{-2t}\]

Solution particulière à l'équation :
\[y_{1_P}(t)=e^{-1000t}\]

D'où :
\[y_1(t)=e^{-2t}+e^{-1000t}=e^{-2t}+y_2(t)\]

Lors de nos tests, on a directement remarqué un problème dû à la stabilité de la méthode. En effet, avec un pas trop grand, nos solutions explosaient. Nous avons dû prendre un pas de l'ordre du dix millième pour avoir une solution correcte. Voici deux tracés des solutions obtenues : le premier avec un pas de 0.1, le deuxième avec un pas de 0.0000001 :

\begin{center}
\includegraphics{../exemple4/0-1.eps}
\end{center}

\begin{center}
\includegraphics{../exemple4/1.eps}
\end{center}

\begin{itemize}
	\item La méthode ne paraît donc pas trop adaptée pour cette équation.
	\item Quand on obtient des points qui semblent corrects, la courbe n'est pas très exploitable. Cela est dû au fait que les abscisses diminuent beaucoup plus lentement que les ordonnées (vu la solution de l'équation), comme on peut le voir sur les deux graphiques suivants :\\
		\includegraphics{../exemple4/2.eps}\\
		\includegraphics{../exemple4/3.eps}
\end{itemize}

\subsection{Cinquième exemple : l'attracteur de Lorenz}

Nous avions un système d'équation de la forme :
\[\left\{ \begin{array}{c c c}
x' &=& -3(x-y)\\
y' &=& -xz + 21x - y\\
z' &=& xy - z
\end{array}\right.\]

\begin{center}
\includegraphics{../exemple5/image.eps}
\includegraphics{../exemple5/image2.eps}

\begin{itemize}
	\item On retrouve bien l'attracteur de Lorenz.
	\item Un premier intérêt est de voir que la méthode d'Adams-Bashforth-2 marche encore très bien pour des systèmes non linéaires.
	\item Le fait qu'on change le pas ne fait pas passer la courbe par les mêmes régions. Le côté chaotique du système explique ces variations. On peut donc soupçonner qu'il existe des méthodes plus adaptées pour les systèmes chaotiques.
\end{itemize}
\end{center}

\subsection{Sixième exemple}

Nous avions un système d'équation de la forme :
\[\left\{ \begin{array}{c c c}
x' &=& \frac{1}{\sqrt{z}+1}\cos(z)\\
y' &=& \frac{1}{\sqrt{z}+1}\sin(z)\\
z' &=& 1
\end{array}\right.\]

\begin{center}
\includegraphics{../exemple6/image3.eps}
\end{center}

\begin{itemize}
	\item L'exemple n'a pas beaucoup d'autre intérêt que de montrer une jolie courbe !
	\item On voit qu'ici, le pas n'influance pas grandement la précision sur la courbe. L'équation est en même temps relativement simple : les dérivées de $x$ et de $y$ ne dépendent que de $z$ à chaque fois, dont sa dérivée vaut 1. 
\end{itemize}



\newpage
\section*{Conclusion}
Ce projet nous aura permis d'implémenter une méthode explicite plus précise que la méthode d'Euler qui est d'ordre 1. On remarque cependant qu'il reste des cas critique (comme l'exemple quatre) ou que pour certains systèmes chaotiques comme l'attracteur de Lorenz, d'autres méthodes peuvent être plus adaptées.\\
Les résultats obtenus restent tout de même très satisfaisants et auront fini de nous convaincre de l'utilité de cette méthode du fait de son implémentation relativement simple. 

\newpage
\appendix
\section{Annexes}
\subsection{Codes supplémentaires}
\subsubsection{Script bash}
\begin{minted}{bash}
	#!/bin/bash


echo "Choisir un exemple."
select choix in `ls | grep "exemple*"`
do
	break
done
cd ${choix}

echo " "
ls|grep "description*" > bidule
dsc=`cat bidule`
rm bidule

dsc=${dsc##*n}
if [ "$dsc" = "1D" ]
then
	cat description1D
elif [ "$dsc" = "2D" ]
then
	cat description2D
else
	cat description3D
fi

echo " "
echo "Que souhaitez-vous faire ?"
echo "1) Modifier le fichier de donnees"
echo "2) Modifier le code source"
echo "3) Lancer le programme"
echo "Autre : quitter le script"
read choix

if [ $choix -eq 1 ]
then
	vim dAdams
	echo " "
elif [ $choix -eq 2 ]
then
	vim adamsBashforth2.c
	echo " "
elif [ $choix -eq 3 ]
then
	gcc adamsBashforth2.c -o adamsBashforth2 -lm
	./adamsBashforth2 < dAdams > res
	
	if [ "$dsc" = "1D" ]
	then
		gnuplot ../1D.g
	elif [ "$dsc" = "2D" ]
	then
		gnuplot ../2D.g
	else
		gnuplot ../3D.g
	fi
	evince image.eps
fi
\end{minted}

\subsubsection{Script gnuplot}
\paragraph{1D.g}
\begin{minted}{gnuplot}
set terminal postscript eps enhanced color
set output "image.eps"
set title "Representation graphique de la solution numerique"
plot "res" u 1:2 w l title "y(t)"
\end{minted}

\paragraph{2D.g}
\begin{minted}{gnuplot}
set terminal postscript eps enhanced color
set output "image.eps"
set title "Representation graphique de la solution numerique"
plot "res" u 2:3 w l title "(x(t),y(t))"
\end{minted}

\paragraph{3D.g}
\begin{minted}{gnuplot}
set terminal postscript eps enhanced color
set output "image.eps"
set title "Representation graphique de la solution numerique"
splot "res" u 2:3:4 w l title "(x(t),y(t),z(t))"
\end{minted}

\end{document}
