\subsection{Deuxième exemple}
L'équation était :
\[xdx - ydy = yx^2dy - xy^2dx\]
\[y(1)=1\]

On transforme l'équation pour avoir :
\[\frac{dy}{dx}=\frac{x(1+y^2)}{y(1+x^2)}\]

Essayons de résoudre l'équation :
\begin{eqnarray*}
	\frac{y}{1+y^2}dy &=& \frac{x}{1+x^2} dx \\
	\frac{1}{2} \ln(1+y^2) &=& \frac{1}{2} \ln(1+x^2) + k \\
	y^2&=&C(1+x^2)-1
\end{eqnarray*}

Or, on veut $y(1)=1$, d'où $1=2C-1 \Rightarrow C=1$\\
On a donc $y^2=x^2$

\begin{center}
\includegraphics{../exemple2/image.eps}
\end{center}

\begin{itemize}
	\item La courbe ressemble beaucoup à ce qu'on doit obtenir (une droite affine). La convergence semble donc assurée.
	\item Avec des pas différents (0.1, 0.01 et 0.0001), la courbe était exactement la même.
\end{itemize}
