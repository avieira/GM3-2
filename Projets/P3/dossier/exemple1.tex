\subsection{Premier exemple}
Dans cet exemple, l'équation était la suivante :
\[y'=\sin(x) - \cos(y), y(0)=1\]



\begin{center}
	\includegraphics{../exemple1/image3.eps}
\end{center}

On remarque tout d'abord que :
\begin{itemize}
	\item On a un certain écart entre un pas de 0.1 et un pas de 0.01 pour la méthode d'Adams-Bashforth-2. Néanmoins, les solutions numériques obetnues semblent converger vers une même représentation graphique. 
	\item Cet exemple illustre aussi bien le fait que la méthode d'Adams-Bashforth soit d'ordre 2 et la méthode d'Euler soit d'ordre 1. En effet, on observe un écart non négligeable entre la méthode d'Euler et de la méthode d'Adams-Bashforth pour un pas de 0.1. Notons toutefois que cet écart diminue avec le pas.
\end{itemize}
