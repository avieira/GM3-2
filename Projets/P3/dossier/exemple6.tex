\subsection{Sixième exemple}

Nous avions un système d'équation de la forme :
\[\left\{ \begin{array}{c c c}
x' &=& \frac{1}{\sqrt{z}+1}\cos(z)\\
y' &=& \frac{1}{\sqrt{z}+1}\sin(z)\\
z' &=& 1
\end{array}\right.\]

\begin{center}
\includegraphics{../exemple6/image3.eps}
\end{center}

\begin{itemize}
	\item L'exemple n'a pas beaucoup d'autre intérêt que de montrer une jolie courbe !
	\item On voit qu'ici, le pas n'influance pas grandement la précision sur la courbe. L'équation est en même temps relativement simple : les dérivées de $x$ et de $y$ ne dépendent que de $z$ à chaque fois, dont sa dérivée vaut 1. 
\end{itemize}
