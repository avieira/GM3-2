\subsection{Exemple introductif}
On commence par un exemple extrêmement simple : 
\[\dot{y}=y\]
\[y(0)=1\]
La solution est ici évidente : $y(x)=e^x$. \\
Cet exemple nous sert surtout à comparer la méthode d'Adams-Bashforth-2 avec celle d'Euler, et de voir la précision qu'on a pour un pas donné. Dans cet exemple, nous avons pris un pas de 0.01.

\begin{center} \includegraphics{../exempleIntro/image.eps} \end{center}

On voit très bien que pour un même pas, la méthode d'Adams-Bashforth-2 est déjà plus précise que celle d'Euler. La courbe obtenue par cette méthode s'approche énormément de la solution exacte, alors qu'on voit déjà que la méthode d'Euler s'en éloigne dans l'intervalle étudié.\\
Voyons maintenant dans d'autres exemples comment fonctionne cette méthode.
