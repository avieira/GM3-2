\section{Présentation du code C}
Tout ce projet a été écrit en C. Nous présentons ici les différents codes écrits pour implémenter les différentes méthodes.\\
Nous avons également écrit un script bash ainsi qu'un script gnuplot pour automatiser le lancement des différents calculs. Ils seront présenter en Annexe.
\subsection{Code pour la méthode d'Euler}
\begin{minted}{c}
void euler(int n,double h,double *V0,double *V1,double *F)
{
	int i;

	for(i=0;i<n;i++)
		V1[i]=V0[i]+h*F[i];
}
\end{minted}

\subsection{Code pour la méthode d'Adams-Bashforth}
\begin{minted}{c}
void adamsBashforth(int n,double h,double *V1,double *V,double *F0,double *F1)
{
	int i;

	for(i=0;i<n;i++)
		V[i]=V1[i]+h/2.*(3*F1[i]-F0[i]);
}
\end{minted}

\subsection{main}
\begin{minted}{c}
void main()
{
	int n;
	double h;
	double *V0;
	double *V1;
	double *V;
	double *F0;
	double *F1;
	double xmax;
	int nmax;
	int i,j;

	//Recuperation des informations generales sur la methode
	scanf("%d %lf %lf",&n,&xmax,&h);

	//Creation des tableaux dynamiques
	V0=(double *) malloc(n*sizeof(double));
	V1=(double *) malloc(n*sizeof(double));
	V=(double *) malloc(n*sizeof(double));
	F0=(double *) malloc(n*sizeof(double));
	F1=(double *) malloc(n*sizeof(double));

	//Lecture et ecriture des conditions initiales
	for(i=0;i<n;i++)
	{
		scanf("%lf",&V0[i]);
		printf("%lf ",V0[i]);
	}
	printf("\n");

	//Calcul du nombre d'iterations
	nmax=trunc((xmax-V0[0])/h);

	//Initialisation avec la methode d'Euler explicite
	f(V0,F0);
	euler(n,h,V0,V1,F0);
	for(i=0;i<n;i++)
		printf("%lf ",V1[i]); 
	printf("\n");

	//Methode d'Adams-Bashforth
	f(V1,F1);
	for(i=2;i<(nmax+1);i++)
	{
		adamsBashforth(n,h,V1,V,F0,F1);
		for(j=0;j<n;j++)
		{
			printf("%lf ",V[j]);
			V0[j]=V1[j];
			V1[j]=V[j];
			f(V0,F0);
			f(V1,F1);
		}
		printf("\n");
	}
}
\end{minted}
