\part{Problème bien posé}

On appelle problème bien posé un problème dont la solution : 
\begin{itemize}
\item existe
\item est unique
\item est stable par rapport aux perturbations
\end{itemize}

\paragraph*{Note sur l'instabilité : \\}
Si la précision sur les solutions est inférieure à la précision qu'on peut avoir tel que le problème est posé, on peut avoir un problème. 

\part{Calcul numérique sur calculateur}
"Approchée" peut venir dedeux facteurs :
\begin{itemize}
	\item Le nombre d'opération à faire est infini, mais on est obligé d'en faire uniquement un nombre fini
	\item Un réel n'a pas de partie finie, et donc, on ne pourra le calculer exactement
\end{itemize}

\bigskip
Représentation d'un réel en machine : \\
$(-1)^s\times f\times 2^e$ avec 
\begin{itemize}
	\item $s=b_{31}$
	\item $e=b_{30}2^7+b_{29}2^6+...+b_{24}2^1+b_{23}-\underbrace{127}_{biais}$
	\item $m=1+b_{22}2^{-1}+b_{21}2^{-2}+...+b_02^{-23}$
\end{itemize}
